\chapter{ISaX}
\index{ISaX}
\label{cha:isax}
\minitoc

ISaX stands for \textbf{interfacing sax} and is a program to transport
information from or to the engine. It is possible to query information
from isax as well as give information to isax which is then able to create
or modify the X11 configuration. If asking isax for data there are two
possible data sources:

\begin{itemize}
\item The auto probed values from the sax registry
\item The current configuration represented by the file
      \textit{/etc/X11/xorg.conf}
\end{itemize}

\section{ISaX Modules}
The option \textbf{-b} is used to obtain data from the sax registry.
If no option is set the current configuration is used. The engine of
SaX is based on seven sections which can be queried and manipulated:

\begin{itemize}
\item \textbf{\underline{Keyboard}}\\
      defines all information about the core keyboard
\item \textbf{\underline{Mouse}}\\
      defines all information about mice
\item \textbf{\underline{Card}}\\
      defines all information about the graphics hardware
\item \textbf{\underline{Desktop}}\\
      defines all information about the desktop which includes
      settings like resolution and colordepth
\item \textbf{\underline{Path}}\\
      defines all information about search paths for fonts
      and special flags for the X-Server
\item \textbf{\underline{Layout}}\\
      defines all information about the server layout which
      includes information about multihead arrangements as well
      as priority lists for keyboard and pointers
\item \textbf{\underline{Extensions}}\\
      defines all information about new X-Server extensions
\end{itemize}

\section{Calling up isax}
\index{ISaX!call up}
Calling up isax does not have to be regulated in a script, but can also be run
by hand. The following command can be used to query data about the graphics
hardware from the current configuration:

\begin{Command}{5cm}
/usr/sbin/isax -l Card
\end{Command}

When obtaining data from the SaX registry the call for graphics hardware
will look like this:

\begin{Command}{5cm}
/usr/sbin/isax -l Card -b
\end{Command}

\section{Creating configurations with isax}
As mentioned in the first section of this chapter isax can be used
to create or modify X11 configurations as well. To do this it is
necessary to specify a so called \textbf{apidata} file. Detailed
information and an example of such a file can be found in
appendix ~\ref{cha:api} (The Variables API File).
The important point here is to mention
that the information for reading and/or writing data with isax provides
a common interface for all operations which can be done with SaX2.
The following command can be used to create a new configuration
from the information specified in the sample file \textit{/var/lib/sax/apidata}:

\begin{Command}{11cm}
/usr/sbin/isax -f /var/lib/sax/apidata -c /tmp/myconfig
\end{Command}

When only modifying based on the currently installed configuration
the command will look like the following. The apidata file in this case
contains only information about the changes which should be migrated
with the current configuration data.

\begin{Command}{11cm}
/usr/sbin/isax -f /var/lib/sax/apidata -c /tmp/myconfig -m
\end{Command}

Both examples will create an output file specified with the option
\textbf{-c}. In this case this results in a file named
\textbf{/tmp/myconfig}.
