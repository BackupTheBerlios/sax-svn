\chapter{Examples of the Problem of Options}
\index{appendix!start examples}
\minitoc

\begin{enumerate}
\item Four cards are inserted, of which the last 3 should be used. For cards 2
  and 4, modules should be set. In this case the command would be:
      \begin{verbatim}
        SaX2 -c 1,2,3 -m 0=mga,2=nv
      \end{verbatim}
      The numbering of the chips begins with 0, as does the order of
      modules. The device 0 is connected to chip 1, device 1 to chip 2, device
      2 to chip 3. 

\item Two cards with a total of 4 chipsets are inserted.
      Three of the 4 chips are on the first card, the other one on the 2nd
      card. A multihead setup should be created which in each case uses the
      first chip on both cards:
      \begin{verbatim}
        SaX2 -c 0 3 
      \end{verbatim}
      If modules need to be allocated for these chipsets, then it should be
      noted that these are detected as card 0 and card 1 and consequently the
      module option needs to be set to 0 and 1: 
      \begin{verbatim}
        SaX -c 0,3 -m 0=mga,1=glint
      \end{verbatim}
\end{enumerate}
