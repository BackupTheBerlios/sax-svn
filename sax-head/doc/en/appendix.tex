\begin{appendix}
%%======================
\chapter{Examples of Using the SaX Batch Mode}
\index{appendix!batch-mode examples}
%%======================
The batch mode in SaX2 allows you to make special settings directly after the 
hardware scan. This mode can be switched on in the \textit{init.pl}
stage. There are two different modes:\\
\begin{itemize}
\item Interactive mode\\
      A shell is provided to enter commands.
\item Profile mode\\
      STDIN is read in and you can specify a profile file which contains shell
      commands.
\end{itemize}
Changes at this point directly influence the contents of the 
\%var hash which is used to construct the registry and create the initial
configuration. It is absolutely essential to understand the hash structure if
you want to use this mode sensibly. 

This structure is not a static form. It can be extended at will with the batch
mode, but it is not clear that all data of the hash can also be used in the
configuration, since simply everything in the hash can be included. The
automatically created structure, when SaX2 is started, is built up as follows:
\\
\begin{verbatim}
#------------------------------------------#
# Files specification...                   #
#------------------------------------------#
Files         ->    0  -> FontPath
Files         ->    0  -> RgbPath
Files         ->    0  -> ModulePath
Files         ->    0  -> LogFile

#-----------------------------------------#
# Module specification...                 #
#-----------------------------------------#
Module        ->    0  -> Load

#-----------------------------------------#
# ServerFlags specification...            #
#-----------------------------------------#
ServerFlags   ->    0  -> Option
ServerFlags   ->    0  -> blank time
ServerFlags   ->    0  -> standby time
ServerFlags   ->    0  -> suspend time
ServerFlags   ->    0  -> off time

#-----------------------------------------#
# Keyboard specification...               #
#-----------------------------------------#
InputDevice   ->    0  -> Identifier
InputDevice   ->    0  -> Driver
InputDevice   ->    0  -> Option            ->  Protocol
InputDevice   ->    0  -> Option            ->  XkbRules
InputDevice   ->    0  -> Option            ->  XkbModel
InputDevice   ->    0  -> Option            ->  XkbLayout
InputDevice   ->    0  -> Option            ->  XkbVariant
InputDevice   ->    0  -> Option            ->  AutoRepeat
InputDevice   ->    0  -> Option            ->  Xleds
InputDevice   ->    0  -> Option            ->  XkbOptions

#-----------------------------------------#
# Mouse specification...                  #
#-----------------------------------------#
InputDevice   ->    1  -> Identifier
InputDevice   ->    1  -> Driver
InputDevice   ->    1  -> Option            ->  Protocol
InputDevice   ->    1  -> Option            ->  Device
InputDevice   ->    1  -> Option            ->  SampleRate
InputDevice   ->    1  -> Option            ->  BaudRate
InputDevice   ->    1  -> Option            ->  Emulate3Buttons
InputDevice   ->    1  -> Option            ->  Emulate3Timeout
InputDevice   ->    1  -> Option            ->  ChordMiddle
InputDevice   ->    1  -> Option            ->  Buttons
InputDevice   ->    1  -> Option            ->  Resolution
InputDevice   ->    1  -> Option            ->  ClearDTR
InputDevice   ->    1  -> Option            ->  ClearRTS
InputDevice   ->    1  -> Option            ->  ZAxisMapping
InputDevice   ->    1  -> Option            ->  MinX
InputDevice   ->    1  -> Option            ->  MaxX
InputDevice   ->    1  -> Option            ->  MinY
InputDevice   ->    1  -> Option            ->  MaxY
InputDevice   ->    1  -> Option            ->  ScreenNumber
InputDevice   ->    1  -> Option            ->  ReportingMode
InputDevice   ->    1  -> Option            ->  ButtonThreshold
InputDevice   ->    1  -> Option            ->  ButtonNumber
InputDevice   ->    1  -> Option            ->  SendCoreEvents

#-----------------------------------------#
# Monitor specification...                #
#-----------------------------------------#
Monitor       ->    0  -> Identifier
Monitor       ->    0  -> VendorName
Monitor       ->    0  -> ModelName
Monitor       ->    0  -> HorizSync
Monitor       ->    0  -> VertRefresh
Monitor       ->    0  -> Modeline          ->  0                -> 640x480
Monitor       ->    0  -> Option

#-----------------------------------------#
# Device specification...                 #
#-----------------------------------------#
Device        ->    0  -> Identifier
Device        ->    0  -> VendorName
Device        ->    0  -> BoardName
Device        ->    0  -> Videoram
Device        ->    0  -> Driver
Device        ->    0  -> Chipset
Device        ->    0  -> Clocks
Device        ->    0  -> BusID
Device        ->    0  -> Option
Device        ->    0  -> Special           ->  hw_cursor

#-----------------------------------------#
# Screen specification...                 #
#-----------------------------------------#
Screen        ->    0  -> Identifier
Screen        ->    0  -> Device
Screen        ->    0  -> Monitor
Screen        ->    0  -> DefaultDepth
Screen        ->    0  -> Depth             ->  8                -> Modes
Screen        ->    0  -> Depth             ->  8                -> ViewPort
Screen        ->    0  -> Depth             ->  8                -> Virtual
Screen        ->    0  -> Depth             ->  8                -> Visual
Screen        ->    0  -> Depth             ->  8                -> Weight
Screen        ->    0  -> Depth             ->  8                -> Black
Screen        ->    0  -> Depth             ->  8                -> White
Screen        ->    0  -> Depth             ->  8                -> Option

#-----------------------------------------#
# ServerLayout specification...           #
#-----------------------------------------#
ServerLayout  ->  all  -> Identifier
ServerLayout  ->  all  -> InputDevice       ->  0                -> id
ServerLayout  ->  all  -> InputDevice       ->  0                -> usage
ServerLayout  ->  all  -> InputDevice       ->  1                -> id
ServerLayout  ->  all  -> InputDevice       ->  1                -> usage
ServerLayout  ->  all  -> Screen            ->  0                -> id
ServerLayout  ->  all  -> Screen            ->  0                -> top
ServerLayout  ->  all  -> Screen            ->  0                -> bottom
ServerLayout  ->  all  -> Screen            ->  0                -> left
ServerLayout  ->  all  -> Screen            ->  0                -> right
\end{verbatim} 


\section{Interactive Mode}
The interactive mode provides the user with the following commands:
\begin{itemize}
\item  \textbf{list}\\
       The \textit{list} command lists all settings of the current registry.

\item  \textbf{see}\\
       The \textit{see} command allows a certain setting to be displayed. For
       example, to see the modules used: 
       \verb+see Module->0->Load+.

\item  \textbf{calc}\\
       The \textit{calc} command lets you calculate modeline timings. For
       example:\\
       \verb+calc 1024x768->70+ calculates a modeline for the mode
       1024x768 at 70 Herz.

\item  \textbf{abort}\\
       Ends interactive mode and discards all changes.

\item  \textbf{exit}\\
       Ends interactive mode and saves all changes.

\item  \textbf{Setting variables}\\
       Setting variables is done by setting the full variable path, by
       including a value allocation. For example:\\
       \verb+Module->0->Load = glx,dri+.
\end{itemize}

\section{Profile Mode and Creating Profile Files}
In some very specific cases it may be necessary for a profile for a card to be
created. SaX2 provides a mechanism which allows you to include known profiles
in the SaX2 package. These profile files are located under:\\
\begin{itemize}
\item /usr/share/sax/profile/
\end{itemize}
If there is an entry in
\textit{/usr/share/sax/sysp/modules/maps/Identity.map} 
which starts with PROFILE=... then the profile for this card is
integrated. The profile files essentially consist of variable values, as the
following example illustrates:
\begin{verbatim}
Screen ->[X]->DefaultDepth         = 24
Monitor->[X]->Modeline->0->640x480 = 36.00 640 680 760 768 480 490 497 520
Monitor->[X]->Modeline->1->800x600 = 49.50 800 856 992 1000 600 612 619 651
\end{verbatim}
Since it is never known in advance for which monitor, screen or device the new
setting is to be made, it is possible to set a place holder in the form of an  
\textbf{[X]} mark, otherwise a number must be entered at this point. 


%%======================
\chapter{Examples of the Problem of Options}
\index{appendix!start examples}
%%======================

\begin{enumerate}
\item Four cards are inserted, of which the last 3 should be used. For cards 2
  and 4, modules should be set. In this case the command would be:
      \begin{verbatim}
        SaX2 -c 1,2,3 -m 0=mga,2=nv
      \end{verbatim}
      The numbering of the chips begins with 0, as does the order of
      modules. The device 0 is connected to chip 1, device 1 to chip 2, device
      2 to chip 3. 

\item Two cards with a total of 4 chipsets are inserted.
      Three of the 4 chips are on the first card, the other one on the 2nd
      card. A multihead setup should be created which in each case uses the
      first chip on both cards:
      \begin{verbatim}
        SaX2 -c 0 3 
      \end{verbatim}
      If modules need to be allocated for these chipsets, then it should be
      noted that these are detected as card 0 and card 1 and consequently the
      module option needs to be set to 0 and 1: 
      \begin{verbatim}
        SaX -c 0,3 -m 0=mga,1=glint
      \end{verbatim}
\end{enumerate}


\end{appendix}

%%% Local Variables: 
%%% mode: latex
%%% TeX-master: t
%%% End: 

