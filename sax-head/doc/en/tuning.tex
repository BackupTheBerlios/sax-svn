\chapter{Xfine tuning}
\label{cha:xfi}
\minitoc
XFine  in SaX2 represents both a module and an independent X11
application. The module \textbf{XFineControl.pm} is used within SaX2 to save
changes in the image geometry and to write these to the configuration file.

The \textbf{xfine.pl} main script writes this change information to the image
geometry as a file in the directory:
\begin{itemize}
\item /var/cache/xfine
\end{itemize}
Per resolution a file is created with change information. The files are named
according to the \textit{SCREEN:XxY} convention.
The format of the files has the following convention:
\begin{verbatim}
   SCREEN:OLDMODE:NEWMODE:DACSPEED
\end{verbatim}

When using XFineControl.pm, there are two different modes:
\begin{itemize}
\item \textbf{Using from within SaX2:}\\
      The main task in using XFineControl.pm is in the creation of the
      so-called \textit{tune} hash. This hash serves in SaX2 as a reference
      for already changed modelines and is checked with each test run. It
      contains the original modeline, the last changed modeline and the
      current modeline. By means of the timing values and the number of
      original modelines, a check is made on whether the tune hash needs to be
      newly created, or if it can serve as a reference. 

\item \textbf{Using from within XFine in local mode:}\\
      If XFine is used as an independent tool, then the tune hash is created
      in advance from a configuration file given as a reference. The tool then
      works on this data in the same way as it would if used from within SaX2.
\end{itemize}

XFine has the following options:
\begin{itemize}
\item \verb+-d | --display+\\
      With this option the display is set in which 
      XFine should be started.
\item \verb+-l | --local+\\
      With this option XFine is started in local mode.
      This means that it will work on a reference to be specified. The
      reference configuration here is also changed.
\item \verb+-c | --config+\\
      With this option the name of the reference configuration is determined. 
      If this is not specified, then /etc/X11/xorg.conf will be used.
\item \verb+-q | --quiet+\\
      This option includes all change information from 
      /var/cache/xfine/ into the specified reference configuration. The
      option  -q makes the option -l necessary.
      After this action,  XFine is ended without access being made to the 
      X display.
\end{itemize}

%%% Local Variables: 
%%% mode: latex
%%% TeX-master: t
%%% End: 

