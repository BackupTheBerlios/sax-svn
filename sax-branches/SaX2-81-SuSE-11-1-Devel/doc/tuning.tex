\chapter{XFine tuning}
\label{cha:xfi}
\minitoc
XFine in SaX2 represents both a module and an independent X11
application. The module \textbf{XFineControl.pm} is used within SaX2 to save
changes in the image geometry and to write these to the configuration file.

\section{The XFine cache}
The \textbf{xfine} application writes this change information to the image
geometry as a file in the directory:
\begin{itemize}
\item /var/cache/xfine
\end{itemize}
Per resolution a file is created with change information. The files are named
according to the \textit{SCREEN:XxY} convention.
The format of the files has the following convention:
\begin{verbatim}
    SCREEN:OLDMODE:NEWMODE:DACSPEED
\end{verbatim}

When using xfine as a standalone application no changes are made
to the actual configuration file, only if used within SaX2 the data written
to \textit{/var/cache/xfine} is processed.

\section{The XFine tune hash}
When SaX2 is using the cache data written by xfine the
main task in using XFineControl.pm is in the creation of the
so-called \textit{tune} hash. This hash serves in SaX2 as a reference
for already changed modelines and is checked with each test run. It
contains the original modeline, the last changed modeline and the
current modeline. By means of the timing values and the number of
original modelines, a check is made on whether the tune hash needs to be
newly created, or if it can serve as a reference.

\newpage

\subsection{The \textit{ImportXFineCache} flag}
To activate processing the cache data and creating the tune hash the
following flag has to be set in the Desktop section of the ISaX
interface file:

\begin{verbatim}
Desktop {
    0 ImportXFineCache = yes
    ...
}
\end{verbatim}

If using libsax the xfine cache is automatically included after a
successful test of the X server. When testing the script
\textit{/var/lib/sax/createTST.pl} is called which will start the
server and the tuning tool.
