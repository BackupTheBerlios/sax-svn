\chapter{XFine tuning}
\label{cha:xfi}
\minitoc
XFine2 ist ein SaX2 Programm-Modul mit dem die Bildgeometrie
in Lage und Gr"o"se angepasst werden kann. Die "Anderungen pro
Aufl"osung (Modeline) werden dabei in einer Datei im Verzeichnis:
\begin{itemize}
\item /var/cache/xfine
\end{itemize}
gehalten. Das Einlesen und verarbeiten der Modeline "Anderungsinformation
wird durch das ISaX Modul implementiert. Ist die Variable 
\textbf{ImportXFineCache} in der ISaX Eingabedatei gesetzt so werden
die Daten in \textit{/var/cache/xfine} verarbeitet und in
der XFree86 Konfigurationsdatei \textbf{/etc/X11/XF86Config} gespeichert.
XFine2 wird aus \textbf{xapi} heraus gestartet.

\section{Name und Format der "Anderungsdateien}
Pro Aufl"osung wird eine Datei mit "Anderungsinformationen erzeugt.
Die Dateinamen unterliegen der Konvention \textit{SCREEN:XxY}.
Das Format der Dateien unterliegt folgender Konvention:
\begin{verbatim}
   SCREEN:OLDMODE:NEWMODE:DACSPEED
\end{verbatim}

Das Einlesen und Verarbeiten der Informationen ist im Modul
\textbf{XFineControl.pm} implementiert. Das Kernst"uck in der 
Verwendung von XFineControl.pm liegt in der Erzeugung des 
sogenannten \textit{tune} Hashes. Dieser Hash dient 
als Referenz f"ur ge"anderte Modelines. Der \textit{tune} Hash wird
aus einer als Referenz angegebenen Konfigurationsdatei heraus
erzeugt und enth"alt die original Modeline, die zuletzt ge"anderte 
Modeline und die aktuelle Modeline. XFine2 kann bei der Verwendung
von SaX2 oder YaST2 in folgenden F"allen zum Einsatz kommen:
\begin{itemize}
\item Bei der Rekonfiguration einer bereits aktiven X11 Umgebung.
\item Beim Test einer ge"anderten X11 Konfiguration
\end{itemize}
Als Referenzkonfiguration dient in beiden F"allen die zuvor
erzeuge Konfigurationsdatei\\
\textit{/var/cache/sax/files/XF86Config}
die beim Speichern nach \textit{/etc/X11/XF86Config} kopiert wird.
Die Referenzkonfiguration wird "uber das Skript \textbf{getconfig}
erzeugt. Dieses Skript erzeugt aus den GUI Datendateien im Verzeichnis
\textit{/var/cache/sax/files/} eine ISaX Eingabedatei und ruft den
ISaX mit dieser Datei auf. Entsprechend des Return-Codes von XFine2 
f"ugt \textbf{xapi} nur noch die Zeile:
\begin{verbatim}
 0 : ImportXFineCache : yes
\end{verbatim}
in die Datei: \textit{/var/cache/sax/files/desktop} ein.
Anschlie"send wird erneut \textbf{getconfig} aufgerufen und damit werden
die Modeline-"Anderungsinformationen dauerhaft in der XF86Config
gespeichert.
