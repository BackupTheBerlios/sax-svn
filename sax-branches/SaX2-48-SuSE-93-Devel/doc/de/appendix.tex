\begin{appendix}
%%======================
\chapter{Beispiele zur Optionsproblematik}
\index{Anhang!Startbeispiele}
%%======================

\begin{enumerate}
\item Eingebaut sind 4 Karten von denen die letzten 3 benutzt 
      werden sollen. F"ur die Karten 2 und 4 sollen module 
      gesetzt werden. In diesem Fall lautet der Aufruf:
      \begin{verbatim}
        SaX2 -c 1,2,3 -m 0=mga,2=nv
      \end{verbatim}
      Die Nummerierung der Chips beginnt bei 0 ebenso die
      Folge der Module. Das Device 0 wird an den Chip 1 gebunden
      Device 1 an den Chip 2 Device 2 an den Chip 3

\item Eingebaut sind zwei Karten mit insgesammt 4 Chipsets.
      3 der vier Chips sind auf der ersten Karte das andere auf der
      zweiten. Es soll ein multihead setup erstellt werden das jeweils
      den ersten Chip auf beiden Karten benutzt:
      \begin{verbatim}
        SaX2 -c 0 3 
      \end{verbatim}
      Sollen f"ur diese Chipsets noch module vergeben werden, so ist 
      zu beachten dass diese als Card 0 und Card 1 erkannt werden und 
      folglich die Moduloption auf 0 und 1 zu setzen ist: 
      \begin{verbatim}
        SaX -c 0,3 -m 0=mga,1=glint
      \end{verbatim}
\end{enumerate}


\end{appendix}

