\chapter{Examples of Using the SaX2 Batch Mode}
\index{appendix!batch-mode examples}
\minitoc

The batch mode in SaX2 allows you to make special settings directly after the 
hardware scan. This mode can be switched on in the \textit{init.pl}
stage. There are two different modes:\\
\begin{itemize}
\item Interactive mode\\
      A shell is provided to enter commands.
\item Profile mode\\
      STDIN is read in and you can specify a profile file which contains shell
      commands.
\end{itemize}
Changes at this point directly influence the contents of the 
\%var hash which is used to construct the registry and create the initial
configuration. It is absolutely essential to understand the hash structure if
you want to use this mode sensibly. 

This structure is not a static form. It can be extended at will with the batch
mode, but it is not clear that all data of the hash can also be used in the
configuration, since simply everything in the hash can be included. The
automatically created structure, when SaX2 is started, is built up as follows:
\\
\begin{verbatim}
#------------------------------------------#
# Files specification...                   #
#------------------------------------------#
Files         ->    0  -> FontPath
Files         ->    0  -> RgbPath
Files         ->    0  -> ModulePath
Files         ->    0  -> LogFile

#-----------------------------------------#
# Module specification...                 #
#-----------------------------------------#
Module        ->    0  -> Load

#-----------------------------------------#
# ServerFlags specification...            #
#-----------------------------------------#
ServerFlags   ->    0  -> Option
ServerFlags   ->    0  -> blank time
ServerFlags   ->    0  -> standby time
ServerFlags   ->    0  -> suspend time
ServerFlags   ->    0  -> off time

#-----------------------------------------#
# Keyboard specification...               #
#-----------------------------------------#
InputDevice   ->    0  -> Identifier
InputDevice   ->    0  -> Driver
InputDevice   ->    0  -> Option            ->  Protocol
InputDevice   ->    0  -> Option            ->  XkbRules
InputDevice   ->    0  -> Option            ->  XkbModel
InputDevice   ->    0  -> Option            ->  XkbLayout
InputDevice   ->    0  -> Option            ->  XkbVariant
InputDevice   ->    0  -> Option            ->  AutoRepeat
InputDevice   ->    0  -> Option            ->  Xleds
InputDevice   ->    0  -> Option            ->  XkbOptions

#-----------------------------------------#
# Mouse specification...                  #
#-----------------------------------------#
InputDevice   ->    1  -> Identifier
InputDevice   ->    1  -> Driver
InputDevice   ->    1  -> Option            ->  Protocol
InputDevice   ->    1  -> Option            ->  Device
InputDevice   ->    1  -> Option            ->  SampleRate
InputDevice   ->    1  -> Option            ->  BaudRate
InputDevice   ->    1  -> Option            ->  Emulate3Buttons
InputDevice   ->    1  -> Option            ->  Emulate3Timeout
InputDevice   ->    1  -> Option            ->  ChordMiddle
InputDevice   ->    1  -> Option            ->  Buttons
InputDevice   ->    1  -> Option            ->  Resolution
InputDevice   ->    1  -> Option            ->  ClearDTR
InputDevice   ->    1  -> Option            ->  ClearRTS
InputDevice   ->    1  -> Option            ->  ZAxisMapping
InputDevice   ->    1  -> Option            ->  MinX
InputDevice   ->    1  -> Option            ->  MaxX
InputDevice   ->    1  -> Option            ->  MinY
InputDevice   ->    1  -> Option            ->  MaxY
InputDevice   ->    1  -> Option            ->  ScreenNumber
InputDevice   ->    1  -> Option            ->  ReportingMode
InputDevice   ->    1  -> Option            ->  ButtonThreshold
InputDevice   ->    1  -> Option            ->  ButtonNumber
InputDevice   ->    1  -> Option            ->  SendCoreEvents

#-----------------------------------------#
# Monitor specification...                #
#-----------------------------------------#
Monitor       ->    0  -> Identifier
Monitor       ->    0  -> VendorName
Monitor       ->    0  -> ModelName
Monitor       ->    0  -> HorizSync
Monitor       ->    0  -> VertRefresh
Monitor       ->    0  -> Modeline          ->  0                -> 640x480
Monitor       ->    0  -> Option

#-----------------------------------------#
# Device specification...                 #
#-----------------------------------------#
Device        ->    0  -> Identifier
Device        ->    0  -> VendorName
Device        ->    0  -> BoardName
Device        ->    0  -> Videoram
Device        ->    0  -> Driver
Device        ->    0  -> Chipset
Device        ->    0  -> Clocks
Device        ->    0  -> BusID
Device        ->    0  -> Option
Device        ->    0  -> Special           ->  hw_cursor

#-----------------------------------------#
# Screen specification...                 #
#-----------------------------------------#
Screen        ->    0  -> Identifier
Screen        ->    0  -> Device
Screen        ->    0  -> Monitor
Screen        ->    0  -> DefaultDepth
Screen        ->    0  -> Depth             ->  8                -> Modes
Screen        ->    0  -> Depth             ->  8                -> ViewPort
Screen        ->    0  -> Depth             ->  8                -> Virtual
Screen        ->    0  -> Depth             ->  8                -> Visual
Screen        ->    0  -> Depth             ->  8                -> Weight
Screen        ->    0  -> Depth             ->  8                -> Black
Screen        ->    0  -> Depth             ->  8                -> White
Screen        ->    0  -> Depth             ->  8                -> Option

#-----------------------------------------#
# ServerLayout specification...           #
#-----------------------------------------#
ServerLayout  ->  all  -> Identifier
ServerLayout  ->  all  -> InputDevice       ->  0                -> id
ServerLayout  ->  all  -> InputDevice       ->  0                -> usage
ServerLayout  ->  all  -> InputDevice       ->  1                -> id
ServerLayout  ->  all  -> InputDevice       ->  1                -> usage
ServerLayout  ->  all  -> Screen            ->  0                -> id
ServerLayout  ->  all  -> Screen            ->  0                -> top
ServerLayout  ->  all  -> Screen            ->  0                -> bottom
ServerLayout  ->  all  -> Screen            ->  0                -> left
ServerLayout  ->  all  -> Screen            ->  0                -> right
\end{verbatim} 


\section{Interactive Mode}
The interactive mode provides the user with the following basic grammar:

\begin{Command}{12cm}
\begin{verbatim}
key -> key ->...-> key [ = ( value | ${variable} ) ]
$variable = ( string | <key sequence> )
\end{verbatim}
\end{Command}

\begin{itemize}
\item  \textbf{!help}\\
       The \textit{!help} command prints a small help text.

\item  \textbf{!remove <sequence>}\\
       The \textit{!remove} command removes the given sequence
       which is a key path to a specific value or subtree. In case of
       a subtree the tree will be removed. In case of a value the value
       is set to an empty string

\item  \textbf{exit}\\
       Ends interactive mode and saves all changes.
\end{itemize}

\section{Profile Mode and Creating Profile Files}
In some very specific cases it may be necessary for a profile for a card to be
created. SaX2 provides a mechanism which allows you to include known profiles
in the SaX2 package. These profile files are located under:\\
\begin{itemize}
\item /usr/share/sax/profile/
\end{itemize}
If there is an entry in
\textit{/usr/share/sax/sysp/modules/maps/Identity.map} 
which starts with PROFILE=... then the profile for this card is
integrated. The profile files essentially consist of variable values, as the
following example illustrates:
\begin{verbatim}
# /.../
# set some fixed values
# ----
Device  -> [X] ->  Driver = fglrx
Desktop -> [X] ->  CalcModelines = no
Monitor -> [X] ->  CalcAlgorithm = XServerPool

# /.../
# save the 16bit Modes into a variable
$Modes  = Screen  -> [X] -> Depth -> 16 -> Modes

# /.../
# save first 16bit mode as 24bit mode
# ----
Screen  -> [X] -> Depth -> 24 -> Modes = ${Modes[0]}

# /.../
# copy 16bit modes to 8bit modes
# ----
Screen  -> [X] -> Depth -> 8 -> Modes  = ${Modes}

# /.../
# remove all Modelines from the Monitor section
# ----
!remove Monitor -> [X] -> Modeline
\end{verbatim}
Since it is never known in advance for which monitor, screen or device the new
setting is to be made, it is possible to set a place holder in the form of an  
\textbf{[X]} mark, otherwise a number must be entered at this point.
