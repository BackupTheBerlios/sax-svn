\chapter{SaX2 Profile}
\label{cha:pro}
\minitoc
Ein Profil ist eine spezielle Beschreibung bestimmter
Kartenspezifischer Eigenschaften die nicht von SaX2 automatisch
erfasst werden k"onnen oder f"ur eine standard Konfiguration 
unbedingt notwendig sind. Ein Profil besteht aus einer oder mehreren 
Profilzeilen die jeweils eine bestimmte Eigenschaft beschreiben. 
Die Formulierung solcher Profilzeilen basiert auf folgendem Format: 

\begin{verbatim}
  Parameter -> Parameter -> ... -> Paramater = Wert
\end{verbatim}

Folgendes ist zu beachten:
\begin{itemize}
\item Gro"s und Klein Schreibung muss beachtet werden
\item Die maximale Schachtelungstiefe liegt bei 6 Parametern
\item Ein leerer Wert enth"alt einfach nichts ( ... Parameter = )
\item Das Trennsymbol zwischen den Paramtern ist der Pfeil ( \verb+->+ )
\end{itemize}

Eingriffe in die Konfiguration durch Profilzeilen beinflussen 
direkt die Inhalte des \%var Hashes der zum Aufbau der Registrierung 
und Erzeugung der Initialkonfiguration verwendet wird. Es ist 
unbedingt erforderlich die Hashstruktur zu kennen um sinnvolle
Profilzeilen zu erzeugen. Die Hash-Struktur ist kein statisches 
Gebilde Sie kann beliebig durch Profilzeilen erweitert werden, 
es ist aber nicht sicher, dass alle Daten des Hashes dann auch in 
der Konfiguration Anwendung finden, da eben einfach alles in den 
Hash eingebracht werden kann. Die automatisch erzeugte Hash-Struktur
beim Start von SaX2 zeigt den prinzipiellen Aufbau des Hashes 
und hilft bei der Erstellung von Profilzeilen. Einen Einblick 
in die Struktur erhalten Sie durch die Daten im Anhang A.\\ 

Da nun das Format der Profilzeilen klar ist gilt es die 
unterschiedlichen Modi zu unterscheiden. Profilzeilen k"onnen zum 
einen in \textbf{Profildateien} eingetragen sein und zum 
anderen "uber die \textbf{SaX2 Shell} innerhalb des \textit{SaX2 batch modus}
eingegeben werden.

\section{Die SaX2 Shell}
Der interaktive Modus bietet dem Nutzer folgende Befehle an:
\begin{itemize}
\item  \textbf{list}\\
       Das \textit{list} Kommando gibt alle Einstellungen der
       aktuellen Registry aus

\item  \textbf{see}\\
       Das \textit{see} Kommando dient der Anzeige einer
       bestimmten Einstellung. Zum Beispiel die verwendeten Module
       \verb+see Module->0->Load+

\item  \textbf{calc}\\
       Das \textit{calc} Kommando dient zur Errechnung von
       Modelinetimings. Zum Beispiel:\\
       \verb+calc 1024x768->70+ Errechnet eine Modeline f"ur
       den Modus 1024x768 bei 70 Herz.

\item  \textbf{abort}\\
       Beendet den interactiven Modus und verwirft alle "Anderungen

\item  \textbf{exit}\\
       Beendet den interactiven Modus und speicher alle "Anderungen

\item  \textbf{Setzen von Variablen}\\
       Das Setzen von Variablen erfolgt durch Eingabe des vollen
       Variablenpfades inklusive einer Wertzuweisung. Zum Beispiel:\\
       \verb+Module->0->Load = glx,dri+
\end{itemize}

\section{Die Erstellung von Profildateien}
Bei den Profildateien muss unterschieden werden zwischen jenen 
die einer bestimmten Karte zugeordnet wurden und von SaX2 selbst 
erkannt und angewendet wurden \textbf{(automatische Profile)} und jenen 
die frei "uber die Option -b <Dateiname> zugeordnet werden 
\textbf{unbound Profile}. F"ur beide Profilarten gelten die nun 
folgenden Bedingungen mit einer Ausnahme:
\begin{itemize}
\item Ungebundene Profile werden immer der Karte \textbf{0} zugeordnet.
      Automatische Profile hingegen kennen die Nummer der Karte an die
      sie gebunden werden sollen. Der Platzhalter \verb+[X]+ ist 
      gleichzusetzen mit \textbf{0}
\item Alle automatischen Profile m"ussen unter /usr/X11R6/lib/sax/profile/ 
      abgelegt werden. Gibt es in der 
      \textit{/usr/X11R6/lib/sax/sysp/modules/maps/Identity.map} einen 
      Eintrag PROFILE=... so wird das Profil f"ur diese Karte 
      automatisch hinzugebunden.
\end{itemize} 

Die Verarbeitung der Profildateien bietet etwas mehr Flexibilit"at 
gegen"uber dem interaktiven Modus. Innerhalb der Profildatei ist
es m"oglich auf bereits definierte Hash-Variablen zuzugreifen sowie
Platzhalter zur Kartenindizierung festzulegen und mit diesen zu 
Rechnen ( einfach Addition ).
\begin{itemize}
\item \textbf{Platzhalter:}\\
      Der Platzhalter darf nur innerhalb der Parameterdefinition 
      auftauchen und hat folgendes Struktur: \verb+[X]+. Dieser
      Konstrukt wird dann zu einer Nummer die der Karte entspricht 
      an die die Profilzeile zugeordnet werden soll. Das ganze
      macht nur Sinn wenn es sich um ein automatisches Profil handelt.
\item \textbf{Indexrechnung:}\\
      Durch die Angabe von \verb=[X+n]= kann
      die durch den Platzhalter festgelegte Nummer um \textit{n}
      vergr"ossert werden.
\item \textbf{Variablen:}\\
      Durch die Angaben von:
      \begin{verbatim}
      ... = {Parameter -> Parameter -> ... -> Parameter}
      \end{verbatim}
      kann der Inhalt dieses Hashwertes genutzt werden. 
\end{itemize}

Diese Eigenschaften sind im interaktiven Modus nicht verf"ugbar.
Das folgende Beispiel enth"alt so ziemlich alles was man mit
einem Profil erreichen kann. Es dient dazu die Dualhead
Unterst"utzung der Matrox G450 zu aktivieren. Diese Karte 
meldet sich leider nicht als Dualheadkarte auf dem PCI/AGP
Bus an, aus diesem Grund muss ein Profil einiges dazu tun.
\begin{verbatim}
 # settings for screen 0
 # ------------------------
 Device  -> [X]   ->  Option           =
 Device  -> [X]   ->  Screen           = 0

 # new device for Screen 1
 # ------------------------
 Device  -> [X+1] ->  BusID            = {Device -> 0 -> BusID}
 Device  -> [X+1] ->  Videoram         = {Device -> 0 -> Videoram}
 Device  -> [X+1] ->  VendorName       = {Device -> 0 -> VendorName}
 Device  -> [X+1] ->  Option           =
 Device  -> [X+1] ->  Identifier       = Device[1]
 Device  -> [X+1] ->  BoardName        = {Device -> 0 -> BoardName}
 Device  -> [X+1] ->  Driver           = {Device -> 0 -> Driver}
 Device  -> [X+1] ->  Screen           = 1

 # new desktop for Screen 1
 # -------------------------
 Screen  -> [X+1] ->  Device           = Device[1]
 Screen  -> [X+1] ->  Monitor          = Monitor[1]
 Screen  -> [X+1] ->  Identifier       = Screen[1]
 Screen  -> [X+1] ->  DefaultDepth     = 16
 Screen  -> [X+1] ->  Depth->16->Modes = 640x480
 Monitor -> [X+1] ->  VertRefresh      = 50-62
 Monitor -> [X+1] ->  ModelName        = Unknown
 Monitor -> [X+1] ->  HorizSync        = 30-32
 Monitor -> [X+1] ->  VendorName       = Unknown
 Monitor -> [X+1] ->  Identifier       = Monitor[1]

 # new Layout settings for multi display mode
 # -------------------------------------------
 ServerLayout -> all -> Screen -> 0 -> left  = <none>
 ServerLayout -> all -> Screen -> 0 -> Right = Screen[1]
 ServerLayout -> all -> Screen -> 1 -> id    = Screen[1]
 ServerLayout -> all -> Screen -> 1 -> left  = Screen[0]
 ServerLayout -> all -> Screen -> 0 -> Right = <none>
\end{verbatim}


