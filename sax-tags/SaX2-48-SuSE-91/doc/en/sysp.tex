\chapter{Sysp}
\index{Sysp}
\label{cha:sys}
\minitoc

Sysp stands for \textbf{system profiler} and is an independent program for 
detecting hardware data.
Sysp is constructed modularly and saves once detected data in so-called perl 
DBM hash files. The data in these files can be read out and processed again
through a sysp call. Detecting and saving information is a central part of
\textit{init.pl}.
This action is carried out once by SaX2 during the initialization. Use is made
of the sysp data, which as a whole makes up the SaX registry, at various points in the configuration process:
\begin{itemize}
\item In \textit{xc.pl} to create the initial configuration.
\item In \textit{xapi.pl} to make data visible in configuration dialogs.
\end{itemize} 
Information which was detected by individual sysp modules is stored in the
directory 
\begin{itemize}
\item /usr/X11R6/lib/sax/sysp/rdbms
\end{itemize}
and re-read from there as well.

\section{Sysp Modules}
\index{Sysp!Module}
In the course of development for SaX2 the following sysp modules were written:
\begin{itemize}
\item \textbf{Keyboard}\\
      This module determines, using the KEYTABLE variable in 
      /etc/rc.config, which type of protocol can be used under X11. With
      systems such as Sparc, for example, a direct hardware scan is started to
      detect this data. 
\item \textbf{Mouse}\\
      This module determines the connection and protocol of all pointer
      devices connected to the system. A condition of this is that these are
      PnP-capable pointer devices, which also provide a checkback signal.
\item \textbf{Server}\\
      This module determines all PCI/AGP graphics cards which are inserted in
      the PCI or AGP bus. Furthermore, it attempts to set an XFree86 4.0
      module allocation by means of the unique vendor and device ID`s of these
      cards, as well as finding special options and extensions. If a single
      ISA card isa used, then it attempts to find out, by registry dump, which
      XFree86 4.0 module can be responsible for this. If there is a mixture of
      AGP, PCI and ISA cards, the ISA cards will definitely not be
      automatically detected.
\item \textbf{Xstuff}\\
      This module collects card-specific data, such as video memory, RamDAC
      speed, possible resolutions per DDC and the synchronization range of the
      monitor in accordance with the resolutions detected. To detect this data
      a minimal XFree86 configuration is created from the result of the
      preceding sysp module, and the X server is started for a brief test
      run. The output of the server is then processed and provides the
      above-mentioned information. The minimum configuration for this server
      start can be found under \textit{/tmp/config.sax}.
\end{itemize}


\section{Calling up sysp}
\index{Sysp!call up}
Calling up sysp does not have to be regulated in a script, but can also be run
by hand. For this case, two modes should be differentiated:
\begin{itemize}
\item Starting a query
\item Starting a hardware detection ( scan )
\end{itemize}
\subsubsection{Sysp Query}
The following command can be used to query data:
\begin{itemize}
\item \verb+sysp -q <Name of the sysp module>+
\end{itemize}

\subsubsection{Sysp Scan}
The following command can be used to start a hardware scan:
\begin{itemize}
\item \verb+sysp -s <Name of the sysp module> [ -o options ]+ 
\end{itemize}
When scanning, it also possible, using the option \verb+-o+ , to pass on
options for the scan. The option list always contains a colon as a separator. 
An example of such a command could look like this:
\begin{itemize}
\item \verb+sysp -s server -o all:0=mga,1=ati+
\end{itemize}

%%% Local Variables: 
%%% mode: latex
%%% TeX-master: t
%%% End: 

