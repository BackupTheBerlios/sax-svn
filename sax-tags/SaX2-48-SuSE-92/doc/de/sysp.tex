\chapter{Sysp}
\index{Sysp}
\label{cha:sys}
\minitoc

Sysp steht fuer \textbf{System profiler} und ist ein 
eigenst"andiges Programm zur Ermittlung von Hardware Daten.
Sysp ist modular aufgebaut und speichert einmal ermittelte
Daten in sogenannten perl DBM hash Dateien ab. Die Daten 
in diesen Dateien koennen durch einen sysp Aufruf wieder
ausgelesen und verarbeitet werden. Das Erkennen und Speichern von 
Hardware-Daten ist zentraler Bestandteil von \textit{init.pl}.
Diese Aktion wird von SaX2 einmalig in der Initialisierung 
vorgenommen. Die Nutzung der sysp Daten die in ihrer Gesammtheit
die SaX2-Registry bilden erfolgt an unterschiedlichen Stellen 
im Konfigurationsablauf:
\begin{itemize}
\item In \textit{xc.pl} um die Initialkonfiguration zu erzeugen.
\item In \textit{xapi} um die Daten in den Konfigurationsdialogen
      verf"ugbar zu machen. Dabei wird ein tool namens \textbf{ISaX}
      benutzt welches wahlweise die Daten aus der automatischen
      Erkennung oder aus der bestehenden X11-Konfigurationsdatei
      in einem einheitlichen Format ausgibt. Weiter Informationen
      zum Theme ISaX sind im Kapitel ~\ref{cha:api} zu finden.
\end{itemize} 
Informationen die durch die einzelnen sysp Module ermittelt wurden 
werden im Verzeichnis
\begin{itemize}
\item /var/cache/sax/sysp/rdbms
\end{itemize}
abgelegt und von dort auch wieder eingelesen. 

\section{Sysp Module}
\index{Sysp!Module}
Im Zuge der SaX2 Entwicklung wurden folgende sysp Module implementiert:
\begin{itemize}
\item \textbf{Keyboard}\\
      Dieses Modul ermittelt anhand der KEYTABLE Variable in der
      \textit{/etc/sysconfig/keyboard} welche Einstellungen zur Abbildung der
      auf der Konsole benutzten Tastatur unter X11 notwendig sind. Dazu 
      wird folgende mapping Datei benutzt: 
      \textit{/usr/X11R6/lib/sax/sysp/maps/Keyboard.map}
      Bei Systemen wie zB sparc wird ein direkter scan der
      Hardware zur Ermittlung dieser Daten gestartet.
\item \textbf{Mouse}\\
      Dieses Modul ermittelt Anschluss und Protokoll aller an das System 
      angeschlossenen Zeigerger"ate. Voraussetzung dafuer ist, dass 
      es sich um ein PNP f"ahiges Zeigerger"at handelt, dass auch eine 
      Rueckmeldung gibt. Die aktuelle Version f"ugt automatisch ein
      USB device ein um beim Einstecken einer USB Maus ohne neu-Konfiguration
      die Unterstuetzung dieser Maus zu gew"ahrleisten.
\item \textbf{Server}\\
      Dieses Modul ermittelt alle PCI/AGP Graphikkarten die auf den 
      PCI oder AGP Bus gesteckt wurden. Weiterhin wird versucht anhand
      der eindeutigen Vendor und Device ID`s dieser Karten eine 
      X11 R6 v4.x Modulzuordnung zu setzen sowie spezielle Optionen 
      und Erweiterungen (Extensions) zu finden. Daten zu erkannten
      Karten werden in der Datei: 
      \textit{/usr/X11R6/lib/sax/sysp/maps/Identity.map} abgelegt.
      Die Datei selbst entsteht aus dem Export einer von SuSE gepflegten
      Datenbank.
\item \textbf{Xstuff}\\
      Dieses Modul ermittelt kartenspezifische Daten wie beispielsweise
      Videospeicher, RamDAC Geschwindigkeit, moegliche Aufloesungen per DDC
      und den Synchronisationsbereich des Monitors entsprechend der erkannten 
      Aufloesungen. Zur Ermittlung dieser Daten wird aus dem Ergebnis der 
      vorangegangenen sysp module eine minimale X11 Konfiguration erstellt
      und der X-Server testweise kurz gestartet. Dieser Serverstart erfolgt nur
      dann wenn mehr als \textbf{eine} Graphikkarte gefunden wurde. Ist
      nur eine Karte im System vorhanden so k"onnen alle notwendigen Daten 
      "uber das VESA BIOS dieser prim"aren Karte ermittelt werden. Bei mehr als
      einer Karte muss jedes nicht prim"are device \textit{soft gebootet}
      werden. Aus diesem Grund wird dann ein X-Server gestartet. Die
      Daten werden nach dem Test durch das Parsen der X-Server Log-Datei
      ermittelt.
\item \textbf{3DLib}\\
      Dieses Modul ermittelt welche Pakete F"ur die 3D Beschleunigung
      installiert sein m"ussen und pr"uft dabei den aktuellen Stand der
      installierten Pakete. Es zeigt an welche Pakete im Bedarfsfall 
      noch installiert werden m"ussen und welches 3D Skript zur Aktivierung
      aufgerufen werden muss. Es wird weiterhin der aktuelle Status des 3D 
      Subsystems angezeigt (Aktiviert/Deaktiviert). Dieses Modul wird 
      keine Pakete installieren, denn diese Aufgabe wird sp"ater durch 
      \textbf{XApI} abgedeckt. N"aheres Dazu ist im OpenGL/3D Workflow;
      Kapitel ~\ref{cha:ogl} zu finden.
\end{itemize}

\section{Aufruf von sysp}
\index{Sysp!Aufruf}
Der Aufruf von sysp ist nicht zwingend an ein Skript gebunden, sondern 
kann auch von Hand erfolgen. Fuer diesen Fall sind zwei Modi zu unterscheiden:
\begin{itemize}
\item Eine Abfrage starten ( Query )
\item Eine Hardwaredetection starten ( Scan )
\end{itemize}
\subsubsection{Sysp Query}
Zur Abfrage von Daten gilt folgender Aufruf:
\begin{itemize}
\item \verb+sysp -q <Name des Sysp Moduls>+
\end{itemize}

\subsubsection{Sysp Scan}
Um einen Hardware scan zu starten gilt folgender Aufruf:
\begin{itemize}
\item \verb+sysp -s <Name des Sysp Moduls> [ Options ]+ 
\end{itemize}
\subsection{Sysp Optionen}
\begin{itemize}
\item \verb+--xbox+\\
Sichert die erkannten Daten nicht als DBM file

\item \verb+--needquestion+\\
Nur m"oglich beim \textit{server} scan um via exit
code zu ermitteln ob die Frage nach 3D gestellt werden wird.
Ist der exit code gleich 1 w"urde eine Frage gestellt werden.

\item \verb+--ask (yes/no)+\\
Vorgabe f"ur die Antwort auf gestellte Fragen

\item \verb+--chip (chiplist)+\\
Benutzt nur die in der komma separierten Liste angegebenen Chips
f"ur den Scan. Beispiel: 0,1

\item \verb+--module (modulelist)+\\
Ordnet den in der komma separierten Liste angegebenen Chips das 
Modul X zu Beispiel: 0=mga,1=nv

\item \verb+--cards+\\
Ausgabe der erkannten Graphikkarte(n) aus

\end{itemize}
