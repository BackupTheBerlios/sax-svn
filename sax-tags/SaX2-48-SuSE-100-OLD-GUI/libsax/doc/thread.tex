\chapter{Thread safety}
%\minitoc

To be thread safe there are a view code points which needs a
locking. The following list describes the serialized lock parts of
the library:

\begin{itemize}
\item library included debug messages to STDERR are embedded into
      flockfile() / funlockfile() calls
\item library initializing calls will lock each other using flock()
\item library exporting code which creates the apidata files
      will apply an flock() during file creation
\end{itemize}

Refering to this the library can be used in a simultaneously
way without crashing and without leaving the configuration in an
inconsistent state. Of course it does not make much sense to
simultaneously configure two different issues in such a case the
last one will win. The following example will demonstrate a thread
example including a simultaneously initialization.

\begin{Command}{12cm}
\begin{small}
\begin{verbatim}
#include <pthread.h>
#include <sax.h>

void* myFunction (void*);

int main (void) {
    pthread_t outThreadID1,outThreadID2;
    pthread_create (&outThreadID1, 0, myFunction, 0);
    pthread_create (&outThreadID2, 0, myFunction, 0);
    pthread_join (outThreadID1,NULL);
    pthread_join (outThreadID2,NULL);
    return 0;
}
void* myFunction (void*) {
    printf ("Checking cache...\n");
    SaXInit* init = new SaXInit;
    if (init->needInit()) {
        printf ("Initialize cache...\n");
        init->doInit();
    }
    printf ("%d : %s\n",
        init->errorCode(),init->errorString().ascii()
    );
    pthread_exit (0);
}
\end{verbatim}
\end{small}
\end{Command}
