\begin{twocolumn}
\setcounter{secnumdepth}{-1}
\chapter{Glossar}

\begin{description}

\begin{small}
%%
\item {\textbf{SaX}\\} {
  Ist ein Oberbegriff f"ur SuSE advanced X-configuration
  SaX ist f"ur XFree86 ab Version 3.3.3 erh"ahltlich.
}

\item {\textbf{Ger"atedatei}\\} {
  Die Schnittstelle zwischen den Funktionen eines Treibers
  und dem Zugriff auf diese \linebreak Funktionen wird durch eine
  Ger"atedatei gebildet. Anhand der Major und Minornummer
  dieser Datei ( auch Node genannt ) erfolgt die Zuordnung 
  zu einem bestimmten Treiber. 
}

\item {\textbf{Ger"at-Node}\\} {
  Ein anderer Ausdruck f"ur \textit{Ger"atedatei}.
}

\item {\textbf{rc.config}\\} {
  Enth"alt Konfigurations- und Startoptionen f"ur s"amtliche Dienste
  des installierten Systems. 
}

\item {\textbf{batchmode}\\} {
  Der Batch Modus steht f"ur den Begriff der Stapelverarbeitung 
  und symbolisiert eine Folge von Aktionen die in Form eines
  Stapels abgearbeitet werden. In SaX2 ist der Batch Modus eine
  Art Kommandointerface in das man Befehle eingeben kann oder 
  Variablen f"ur sp"ateren Gebrauch definieren kann. Der Batch 
  Mode in SaX2 kann sowohl interaktiv oder auch "uber eine Datei
  gesteuert werden. 
}

%%
\end{small}
\end{description}
\end{twocolumn}


\onecolumn

