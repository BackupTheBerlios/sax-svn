\chapter{libsax - Language bindings}
\minitoc

libsax has been developed as a C++ written library which means
including this library into languages providing an object model
is a possible task whereas bindings to non object oriented languages
requires a self written wrapper for accessing object methods.

\section{SWIG}

SWIG (Simple Wrapper Interface Generator) is a software development tool
that simplifies the task of
interfacing different languages to C and C++ programs. In a nutshell,
SWIG is a compiler that takes C declarations and creates the wrappers
needed to access those declarations from other languages
including csharp, java, perl5, php4, python or tcl. Creating language
bindings for libsax requires the following tasks:

\begin{itemize}
\item All libsax C++ header files must be available to create the binding
\item C++ operator support is not available in many scripting languages.
      Because of this reason the binding should disable operator support
      and the library has to provide methods as well as operators.
\item Typemaps for transforming the Qt types used within the library
      must be provided for each destination language.
\item Qts Signal/Slot concept is not supported by the destination languages.
      libsax is using this mechanism to implement exception handling.
      As consequence only the traditional error handling is available for
      the bindings and the SaXException class needs to be wrapped within
	  the bindings interface definition.
\end{itemize}

\newpage

\section{Interface template file for libsax (SaX.i)}

The following template illustrate the steps which needs to be
implemented to be able to map all libsax types into the destination
language.

\begin{Command}{12cm}
\begin{small}
\begin{verbatim}
//==================================
// Interface definition for libsax
//----------------------------------
#define NO_OPERATOR_SUPPORT 1
%module SaX
%{
#include "../sax.h"
%}
//==================================
// SWIG includes
//----------------------------------
%include exception.i

//==================================
// Typemaps
//----------------------------------
//==================================
// Allow QString return types
//----------------------------------
// [ destination language dependant ]
// ...

//==================================
// Allow QString refs as parameters
//----------------------------------
// [ destination language dependant ]
// ...

//==================================
// Allow QDict<QString> return types
//----------------------------------
// [ destination language dependant ]
// ...

//==================================
// Allow QList<QString> return types
//----------------------------------
// [ destination language dependant ]
// ...

\end{verbatim}
\end{small}
\end{Command}

\newpage

\begin{Command}{12cm}
\begin{small}
\begin{verbatim}
//==================================
// Exception class wrapper...
//----------------------------------
class SaXException {
    public:
    int   errorCode          ( void );
    bool  havePrivileges     ( void );
    void  errorReset         ( void );

    public:
    QString errorString  ( void );
    QString errorValue   ( void );

    public:
    void setDebug ( bool = true );
};
//==================================
// ANSI C/C++ declarations...
//----------------------------------
%include "../storage.h"
%include "../process.h"
%include "../export.h"
%include "../import.h"
%include "../init.h"
%include "../config.h"
%include "../card.h"
%include "../keyboard.h"
%include "../pointers.h"
%include "../desktop.h"
%include "../extensions.h"
%include "../layout.h"
%include "../path.h"
%include "../sax.h"
\end{verbatim}
\end{small}
\end{Command}

\newpage

\subsection{Interface explanations}
The interface file is divided into four \textit{sections} which
handles the following interfacing problems:

\begin{enumerate}
\item Module name and include files to be able to create the
      wrapper code. The namespace used here is called \textbf{SaX}
\item type mappings from C++ into the destination language.
      Currently perl,python,java and csharp types are supported.
\item Exception class wrapper which doesn't include the Qt signal/slot
      definitions
\item Declarations used to create an interface for the destination
      language. Refering to perl this information is used to create
      the appropriate SaX.pm file
\end{enumerate}

\section{Example: libsax used with perl...}
The following examples will do the same as the last example from the
Examples chapter; Changing the default colordepth of the current
configuration to 24 bit.

\begin{Command}{12cm}
\begin{small}
\begin{verbatim}
use SaX;
sub main {
    my %section;
    my @importID = (
        $SaX::SAX_CARD, $SaX::SAX_DESKTOP, $SaX::SAX_PATH
    );
    my $config = new SaX::SaXConfig;
    foreach my $id (@importID) {
        $import = new SaX::SaXImport ( $id );
        $import -> setSource ( $SaX::SAX_SYSTEM_CONFIG );
        $import -> doImport();
        $config -> addImport ( $import );
        $section{$import->getSectionName()} = $import;
    }
    my $mDesktop = new SaX::SaXManipulateDesktop (
        $section{Desktop},$section{Card},$section{Path}
    );
    if ($mDesktop->selectDesktop (0)) {
        $mDesktop->setColorDepth (24);
    }
    $config -> setMode (SaX::SAX_MERGE);
    $config -> createConfiguration();
}
main();
\end{verbatim}
\end{small}
\end{Command}
