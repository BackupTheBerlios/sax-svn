\begin{twocolumn}
\setcounter{secnumdepth}{-1}
\chapter{Glossary}

\begin{description}

\begin{small}
%%
\item {\textbf{SaX}\\} {
  Is an abbreviation for SuSE advanced X-configuration
  SaX is available for XFree86 from version 3.3.3 onwards.
}

\item {\textbf{Device File}\\} {
  The interface between the functions of a driver and the access to these
  \linebreak functions is formed by a device file. By means of the major and
  minor number of this file (also called node), allocation is made to a
  specific driver.  
}

\item {\textbf{Device Node}\\} {
  Another term for \textit{device file}.
}

\item {\textbf{rc.config}\\} {
  Contains configuration and start options for all services of the installed
  system. 
}

\item {\textbf{batchmode}\\} {
  The batch mode stands for the concept of batch processing, and symbolizes a
  series of actions which are processed in the form of a stack. In SaX2 the 
  batch mode is a kind of command interface into which you can enter commands
  or define variables for later use. The batch mode in SaX2 can be controlled
  automatically, or via a file.  
}

%%
\end{small}
\end{description}
\end{twocolumn}


\onecolumn

%%% Local Variables: 
%%% mode: latex
%%% TeX-master: t
%%% End: 

